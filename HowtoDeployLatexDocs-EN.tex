\documentclass[]{article}
\usepackage{lmodern}
\usepackage{listings}
\usepackage{color}
\usepackage{amssymb,amsmath}
\usepackage{ifxetex,ifluatex}
\usepackage{fixltx2e}

\definecolor{mygray}{rgb}{0.4,0.4,0.4}
\definecolor{mygreen}{rgb}{0,0.8,0.6}
\definecolor{myorange}{rgb}{1.0,0.4,0}
\definecolor{myblue}{rgb}{0.2,0.4,1}
\definecolor{mylightgray}{rgb}{0.98,0.98,0.98}
\definecolor{mymauve}{rgb}{0.58,0,0.82}

\lstset{
  backgroundcolor=\color{mylightgray},
  basicstyle=\footnotesize,
  breaklines=true,
  captionpos=b,
  commentstyle=\small\color{mygreen},
  deletekeywords={...},
  escapeinside={\%*}{*)},
  extendedchars=true,
  keepspaces=true,
  keywordstyle=\color{myblue},
  otherkeywords={*,os:,language:,sudo:,apt_packages:,script:,after_success:,deploy:,provider:,api_key:,secure:,file:,skip_cleanup:,on:,repo:,...},
  numbers=left,
  numbersep=5pt,
  numberstyle=\tiny\color{mygray},
  showspaces=false,
  showstringspaces=false,
  showtabs=false,
  stepnumber=1,
  stringstyle=\color{myorange},
  tabsize=2,
  title=\lstname
}

\ifnum 0\ifxetex 1\fi\ifluatex 1\fi=0
  \usepackage[T1]{fontenc}
  \usepackage[utf8]{inputenc}
\else
  \ifxetex
    \usepackage{mathspec}
    \usepackage{xltxtra,xunicode}
  \else
    \usepackage{fontspec}
  \fi
  \defaultfontfeatures{Mapping=tex-text,Scale=MatchLowercase}
  \newcommand{\euro}{€}
\fi

\IfFileExists{upquote.sty}{\usepackage{upquote}}{}

\IfFileExists{microtype.sty}{
\usepackage{microtype}
\UseMicrotypeSet[protrusion]{basicmath}
}{}
\ifxetex
  \usepackage[setpagesize=false,
              unicode=false,
              xetex]{hyperref}
\else
  \usepackage[unicode=true]{hyperref}
\fi

\hypersetup{breaklinks=true,
            bookmarks=true,
            pdfauthor={},
            pdftitle={},
            colorlinks=true,
            citecolor=blue,
            urlcolor=blue,
            linkcolor=magenta,
            pdfborder={0 0 0}}
\urlstyle{same}
\setlength{\parindent}{0pt}
\setlength{\parskip}{6pt plus 2pt minus 1pt}
\setlength{\emergencystretch}{3em}
\providecommand{\tightlist}{
  \setlength{\itemsep}{0pt}\setlength{\parskip}{0pt}}
\setcounter{secnumdepth}{0}

\date{}

\ifx\paragraph\undefined\else
\let\oldparagraph\paragraph
\renewcommand{\paragraph}[1]{\oldparagraph{#1}\mbox{}}
\fi
\ifx\subparagraph\undefined\else
\let\oldsubparagraph\subparagraph
\renewcommand{\subparagraph}[1]{\oldsubparagraph{#1}\mbox{}}
\fi

\begin{document}

\section{Deploy LaTeX as PDF via GitHub and
TravisCI}\label{deploy-latex-as-pdf-via-github-and-travisci}

This document will explain how to automatically convert LaTeX documents
into PDFs and can publish them via the services
\href{https://github.com/}{GitHub} and
\href{https://travis-ci.org/}{TravisCI}. The advantage of this method
over the local transformation (Eg via git hooks) and insertion into the
repository, among other things are:

\begin{itemize}
\tightlist
\item
  No binary data in the repo
\item
  Check the code at each commit
\item
  Testing of pull requests
\item
  Independence from the local system
\end{itemize}

Of course, this can be extended to cover the conversion of Markdown by
using \textbf{pandoc}.

In the instructions, I assume that it is a public document. Therefore
free accounts can be used at \href{https://github.com/}{GitHub} and
\href{https://travis-ci.org/}{TravisCI}. There are no costs through the
use.

\subsection{Step by step}\label{step-by-step}

\begin{enumerate}
\def\labelenumi{\arabic{enumi}.}
\tightlist
\item
  Creation of a \textbf{git}-Repository
\end{enumerate}

\begin{itemize}
\tightlist
\item
  \href{https://github.com/SimonWaldherr/golang-examples\#fork-destination-box}{fork
  this repo} \emph{or}
\item
  \href{https://github.com/new}{create a new repo on GitHub}
\end{itemize}

\begin{enumerate}
\def\labelenumi{\arabic{enumi}.}
\tightlist
\item
  \texttt{git\ clone} the repo (\emph{yes, you could also edit online,
  but we ignore it this time})
\item
  for configuration, we use \textbf{Travis.yml-YAML file} (more details
  below)
\end{enumerate}

\begin{itemize}
\tightlist
\item
  in line 16 is sample code to see howto use Markdown as input format
\item
  in line 20 is the call of the \texttt{pdflatex} program for converting
  .tex to .pdf
\item
  in line 23, the PDF gets converted to a
  \href{https://de.wikipedia.org/wiki/DjVu}{DjVu document}
\item
  in line 26 you can see how to transfer the generated data via wget
\item
  the at line 29 defined provider \textbf{releases} stands for
  \textbf{GitHub releases}
\item
  in line 30 and 31 the required \textbf{oauth} string for
  \href{http://docs.travis-ci.com/user/deployment/releases/}{storing
  releases} is saved, it can be generated with different commands

  \begin{itemize}
  \tightlist
  \item
    \texttt{travis\ setup\ releases} generates the full
    \emph{deploy}-block
  \item
    \texttt{travis\ encrypt\ ***\$OAUTH\_STRING***} generates only the
    encrypted part in row 31
  \end{itemize}
\item
  in lines 32 to 34 the files to upload are defined
\end{itemize}

\begin{enumerate}
\def\labelenumi{\arabic{enumi}.}
\tightlist
\item
  to activate the repo for Travis builds it may be necessary to refresh
  the list of repos in your
  \href{https://travis-ci.org/profile}{TravisCI profile}.
\item
  now the LaTeX or Markdown files are to be created (remember to update
  the filenames in \textbf{.travis.yml})
\item
  at every \texttt{git\ push} a GNU/Linux VM will be started
  automatically at TravisCI which generates the PDF and DjVu files and
  publishes them as GitHub release. \emph{This file was created exactly
  this way, each step can be viewed in the
  \href{https://travis-ci.org/SimonWaldherr/testrepo}{TravisCI
  logfile}.}
\end{enumerate}

\subsection{Travis Config (YAML)}\label{travis-config-yaml}

\lstinputlisting[language=PHP]{.travis.yml}

\subsection{Terms of Use}\label{terms-of-use}

All Rights Reserved. No guarantee for completeness or actuality. All
content of this Repository is licensed under a
\href{http://creativecommons.org/licenses/by-sa/4.0/}{Creative Commons
Attribution - Share Alike 4.0 License International}.

\end{document}
