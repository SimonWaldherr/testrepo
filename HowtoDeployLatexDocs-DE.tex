\documentclass[]{article}
\usepackage{lmodern}
\usepackage{listings}
\usepackage{color}
\usepackage{amssymb,amsmath}
\usepackage{ifxetex,ifluatex}
\usepackage{fixltx2e}

\definecolor{mygray}{rgb}{0.4,0.4,0.4}
\definecolor{mygreen}{rgb}{0,0.8,0.6}
\definecolor{myorange}{rgb}{1.0,0.4,0}
\definecolor{myblue}{rgb}{0.2,0.4,1}
\definecolor{mylightgray}{rgb}{0.98,0.98,0.98}
\definecolor{mymauve}{rgb}{0.58,0,0.82}

\lstset{
  backgroundcolor=\color{mylightgray},
  basicstyle=\footnotesize,
  breaklines=true,
  captionpos=b,
  commentstyle=\small\color{mygreen},
  deletekeywords={...},
  escapeinside={\%*}{*)},
  extendedchars=true,
  keepspaces=true,
  keywordstyle=\color{myblue},
  otherkeywords={*,os:,language:,sudo:,apt_packages:,script:,after_success:,deploy:,provider:,api_key:,secure:,file:,skip_cleanup:,on:,repo:,...},
  numbers=left,
  numbersep=5pt,
  numberstyle=\tiny\color{mygray},
  showspaces=false,
  showstringspaces=false,
  showtabs=false,
  stepnumber=1,
  stringstyle=\color{myorange},
  tabsize=2,
  title=\lstname
}

\ifnum 0\ifxetex 1\fi\ifluatex 1\fi=0
  \usepackage[T1]{fontenc}
  \usepackage[utf8]{inputenc}
\else
  \ifxetex
    \usepackage{mathspec}
    \usepackage{xltxtra,xunicode}
  \else
    \usepackage{fontspec}
  \fi
  \defaultfontfeatures{Mapping=tex-text,Scale=MatchLowercase}
  \newcommand{\euro}{€}
\fi

\IfFileExists{upquote.sty}{\usepackage{upquote}}{}

\IfFileExists{microtype.sty}{
\usepackage{microtype}
\UseMicrotypeSet[protrusion]{basicmath}
}{}
\ifxetex
  \usepackage[setpagesize=false,
              unicode=false,
              xetex]{hyperref}
\else
  \usepackage[unicode=true]{hyperref}
\fi

\hypersetup{breaklinks=true,
            bookmarks=true,
            pdfauthor={},
            pdftitle={},
            colorlinks=true,
            citecolor=blue,
            urlcolor=blue,
            linkcolor=magenta,
            pdfborder={0 0 0}}
\urlstyle{same}
\setlength{\parindent}{0pt}
\setlength{\parskip}{6pt plus 2pt minus 1pt}
\setlength{\emergencystretch}{3em}
\providecommand{\tightlist}{
  \setlength{\itemsep}{0pt}\setlength{\parskip}{0pt}}
\setcounter{secnumdepth}{0}

\date{}

\ifx\paragraph\undefined\else
\let\oldparagraph\paragraph
\renewcommand{\paragraph}[1]{\oldparagraph{#1}\mbox{}}
\fi
\ifx\subparagraph\undefined\else
\let\oldsubparagraph\subparagraph
\renewcommand{\subparagraph}[1]{\oldsubparagraph{#1}\mbox{}}
\fi

\begin{document}

\section{Veröffentlichen von LaTeX Dokumenten als PDF mittels GitHub und TravisCI}\label{veruxf6ffentlichen-von-latex-dokumenten-als-pdf-mittels-github-und-travisci}

Dieses Dokument soll erklären, wie man mit den Diensten
\href{https://github.com/}{GitHub} und
\href{https://travis-ci.org/}{TravisCI} LaTeX Dokumente automatisch in
PDFs umzuwandeln und veröffentlichen kann. Der Vorteil dieses Verfahrens
gegenüber der lokalen Umwandlung (bspw. via git hooks) und Inkludierung
in das Repository ist u.a.:

\begin{itemize}
\tightlist
\item
  keine Binärdaten im Repo
\item
  Prüfung des Codes bei jedem Commit
\item
  Prüfung von Pull Requests
\item
  Unabhängigkeit vom lokalem System
\end{itemize}

Natürlich lässt sich dies auch um die Komponente \textbf{pandoc}
erweitern um somit auch die Umwandlung von Markdown abzudecken.

In der Anleitung gehe ich davon aus, dass es sich um ein öffentliches
Dokument handelt. Somit können Gratis Accounts bei
\href{https://github.com/}{GitHub} und
\href{https://travis-ci.org/}{TravisCI} verwendet werden. Es entstehen
keine Kosten durch die Nutzung.

\subsection{Schritt für Schritt}\label{schritt-fuxfcr-schritt}

\begin{enumerate}
\def\labelenumi{\arabic{enumi}.}
\tightlist
\item
  Erstellung eines \textbf{git}-Repositories
\end{enumerate}

\begin{itemize}
\tightlist
\item
  \href{https://github.com/SimonWaldherr/golang-examples\#fork-destination-box}{dieses
  Repo forken} \emph{oder}
\item
  \href{https://github.com/new}{neues Repo auf GitHub erstellen}
\end{itemize}

\begin{enumerate}
\def\labelenumi{\arabic{enumi}.}
\tightlist
\item
  \texttt{git\ clone} das Repo (\emph{ja, man könnte auch online
  editieren, aber das ignorieren wir mal})
\item
  zur Konfiguration benutzen wir die \textbf{.travis.yml-YAML-Datei}
  (genaueres weiter unten)
\end{enumerate}

\begin{itemize}
\tightlist
\item
  in Zeile 16 ist Beispielcode zu sehen um Markdown als Ausgangsformat
  zu verwenden
\item
  in Zeile 20 ist der Aufruf des \texttt{pdflatex} Programms zur
  Umwandlung von .tex zu .pdf
\item
  in Zeile 23 wird das PDF in ein
  \href{https://de.wikipedia.org/wiki/DjVu}{DjVu-Dokument} umgewandelt
\item
  in Zeile 26 ist zu sehen, wie man mittels wget die generierten Daten
  überträgt
\item
  der in Zeile 29 definierte provider \textbf{releases} steht für
  \textbf{GitHub Releases}
\item
  in Zeile 30 und 31 ist der \textbf{oauth} String der zur
  \href{http://docs.travis-ci.com/user/deployment/releases/}{Speicherung
  der Releases} benötigt wird gespeichert, dies kann über verschiedene
  Befehle erzeugt werden

  \begin{itemize}
  \tightlist
  \item
    \texttt{travis\ setup\ releases} generiert den ganzen
    \emph{deploy}-Block
  \item
    \texttt{travis\ encrypt\ ***\$OAUTH\_STRING***} erzeugt nur den
    verschlüsselten Teil in Zeile 31
  \end{itemize}
\item
  in den Zeilen 32 bis 34 werden die hochzuladenden Dateien definiert
\end{itemize}

\begin{enumerate}
\def\labelenumi{\arabic{enumi}.}
\tightlist
\item
  um das Repo für Travis Builds zu aktivieren kann es notwendig sein,
  die Liste der Repos im \href{https://travis-ci.org/profile}{TravisCI
  Profil} zu aktualisieren.
\item
  nun sind die LaTeX oder Markdown Dateien zu erstellen (vergiss nicht
  die Dateinamen in \textbf{.travis.yml} zu aktualisieren)
\item
  bei jedem \texttt{git\ push} wird nun automatisch eine GNU/Linux VM
  bei TravisCI gestartet, welche die PDF und DjVu Dateien generiert und
  als GitHub Release veröffentlicht. \emph{Diese Datei wurde genau so
  erstellt, die einzelnen Schritte können im
  \href{https://travis-ci.org/SimonWaldherr/testrepo}{TravisCI Logfile}
  angesehen werden.}
\end{enumerate}

\subsection{Travis Konfig (YAML)}\label{travis-konfig-yaml}

\lstinputlisting[language=PHP]{.travis.yml}


\subsection{Rechtliches}\label{rechtliches}

Alle Angaben ohne Gewähr. Keine Garantie für Vollständigkeit oder
Aktualität. Jeglicher Inhalt dieses Repos ist lizenziert unter einer
\href{http://creativecommons.org/licenses/by-sa/4.0/}{Creative Commons
Namensnennung - Weitergabe unter gleichen Bedingungen 4.0 International
Lizenz}.

\end{document}
