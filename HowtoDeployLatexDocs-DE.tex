\documentclass[]{article}
\usepackage{lmodern}
\usepackage{amssymb,amsmath}
\usepackage{ifxetex,ifluatex}
\usepackage{fixltx2e} % provides \textsubscript
\ifnum 0\ifxetex 1\fi\ifluatex 1\fi=0 % if pdftex
  \usepackage[T1]{fontenc}
  \usepackage[utf8]{inputenc}
\else % if luatex or xelatex
  \ifxetex
    \usepackage{mathspec}
    \usepackage{xltxtra,xunicode}
  \else
    \usepackage{fontspec}
  \fi
  \defaultfontfeatures{Mapping=tex-text,Scale=MatchLowercase}
  \newcommand{\euro}{€}
\fi
% use upquote if available, for straight quotes in verbatim environments
\IfFileExists{upquote.sty}{\usepackage{upquote}}{}
% use microtype if available
\IfFileExists{microtype.sty}{%
\usepackage{microtype}
\UseMicrotypeSet[protrusion]{basicmath} % disable protrusion for tt fonts
}{}
\ifxetex
  \usepackage[setpagesize=false, % page size defined by xetex
              unicode=false, % unicode breaks when used with xetex
              xetex]{hyperref}
\else
  \usepackage[unicode=true]{hyperref}
\fi
\usepackage[usenames,dvipsnames]{color}
\hypersetup{breaklinks=true,
            bookmarks=true,
            pdfauthor={},
            pdftitle={},
            colorlinks=true,
            citecolor=blue,
            urlcolor=blue,
            linkcolor=magenta,
            pdfborder={0 0 0}}
\urlstyle{same}  % don't use monospace font for urls
\setlength{\parindent}{0pt}
\setlength{\parskip}{6pt plus 2pt minus 1pt}
\setlength{\emergencystretch}{3em}  % prevent overfull lines
\providecommand{\tightlist}{%
  \setlength{\itemsep}{0pt}\setlength{\parskip}{0pt}}
\setcounter{secnumdepth}{0}

\date{}

% Redefines (sub)paragraphs to behave more like sections
\ifx\paragraph\undefined\else
\let\oldparagraph\paragraph
\renewcommand{\paragraph}[1]{\oldparagraph{#1}\mbox{}}
\fi
\ifx\subparagraph\undefined\else
\let\oldsubparagraph\subparagraph
\renewcommand{\subparagraph}[1]{\oldsubparagraph{#1}\mbox{}}
\fi

\begin{document}

\section{Veröffentlichen von LaTeX Dokumenten als PDF mittels GitHub und
TravisCI}\label{veruxf6ffentlichen-von-latex-dokumenten-als-pdf-mittels-github-und-travisci}

Dieses Dokument soll erklären, wie man mit den Diensten
\href{https://github.com/}{GitHub} und
\href{https://travis-ci.org/}{TravisCI} LaTeX Dokumente automatisch in
PDFs umzuwandeln und veröffentlichen kann. Der Vorteil dieses Verfahrens
gegenüber der lokalen Umwandlung (bspw. via git hooks) und Inkludierung
in das Repository ist u.a.:

\begin{itemize}
\tightlist
\item
  keine Binärdaten im Repo
\item
  Prüfung des Codes bei jedem Commit
\item
  Prüfung von Pull Requests
\item
  Unabhängigkeit vom lokalem System
\end{itemize}

Natürlich lässt sich dies auch um die Komponente \textbf{pandoc}
erweitern um somit auch die Umwandlung von Markdown abzudecken.

In der Anleitung gehe ich davon aus, dass es sich um ein öffentliches
Dokument handelt. Somit können Gratis Accounts bei
\href{https://github.com/}{GitHub} und
\href{https://travis-ci.org/}{TravisCI} verwendet werden. Es entstehen
keine Kosten durch die Nutzung.

\subsection{Schritt für Schritt}\label{schritt-fuxfcr-schritt}

\begin{enumerate}
\def\labelenumi{\arabic{enumi}.}
\tightlist
\item
  Erstellung eines \textbf{git}-Repositories
\end{enumerate}

\begin{itemize}
\tightlist
\item
  \href{https://github.com/SimonWaldherr/golang-examples\#fork-destination-box}{dieses
  Repo forken} \emph{oder}
\item
  \href{https://github.com/new}{neues Repo auf GitHub erstellen}
\end{itemize}

\begin{enumerate}
\def\labelenumi{\arabic{enumi}.}
\tightlist
\item
  \texttt{git\ clone} das Repo (\emph{ja, man könnte auch online
  editieren, aber das ignorieren wir mal})
\item
  zur Konfiguration benutzen wir die \textbf{.travis.yml-YAML-Datei}
  (genaueres weiter unten)
\end{enumerate}

\begin{itemize}
\tightlist
\item
  in Zeile 16 ist Beispielcode zu sehen um Markdown als Ausgangsformat
  zu verwenden
\item
  in Zeile 20 ist der Aufruf des \texttt{pdflatex} Programms zur
  Umwandlung von .tex zu .pdf
\item
  in Zeile 23 wird das PDF in ein
  \href{https://de.wikipedia.org/wiki/DjVu}{DjVu-Dokument} umgewandelt
\item
  in Zeile 26 ist zu sehen, wie man mittels wget die generierten Daten
  überträgt
\item
  der in Zeile 29 definierte provider \textbf{releases} steht für
  \textbf{GitHub Releases}
\item
  in Zeile 30 und 31 ist der \textbf{oauth} String der zur
  \href{http://docs.travis-ci.com/user/deployment/releases/}{Speicherung
  der Releases} benötigt wird gespeichert, dies kann über verschiedene
  Befehle erzeugt werden

  \begin{itemize}
  \tightlist
  \item
    \texttt{travis\ setup\ releases} generiert den ganzen
    \emph{deploy}-Block
  \item
    \texttt{travis\ encrypt\ ***\$OAUTH\_STRING***} erzeugt nur den
    verschlüsselten Teil in Zeile 31
  \end{itemize}
\item
  in den Zeilen 32 bis 34 werden die hochzuladenden Dateien definiert
\end{itemize}

\begin{enumerate}
\def\labelenumi{\arabic{enumi}.}
\tightlist
\item
  um das Repo für Travis Builds zu aktivieren kann es notwendig sein,
  die Liste der Repos im \href{https://travis-ci.org/profile}{TravisCI
  Profil} zu aktualisieren.
\item
  nun sind die LaTeX oder Markdown Dateien zu erstellen (vergiss nicht
  die Dateinamen in \textbf{.travis.yml} zu aktualisieren)
\item
  bei jedem \texttt{git\ push} wird nun automatisch eine GNU/Linux VM
  bei TravisCI gestartet, welche die PDF und DjVu Dateien generiert und
  als GitHub Release veröffentlicht. \emph{Diese Datei wurde genau so
  erstellt, die einzelnen Schritte können im
  \href{https://travis-ci.org/SimonWaldherr/testrepo}{TravisCI Logfile}
  angesehen werden.}
\end{enumerate}

\subsection{Travis Konfig (YAML)}\label{travis-konfig-yaml}

\begin{verbatim}
os: linux

# we use the R VM because it contains LaTeX and pandoc
language: r

# we need sudo, so we depend on the old travis-ci infrastructure
sudo: required

# we can install aditional software
apt_packages:
# we download pdf2djvu to also generate a djvu file of our LaTeX source
  - pdf2djvu

script:
# if you have a markdown source, you have to uncomment the next line
#  - pandoc -s -o HowtoDeployLatexDocs.tex HowtoDeployLatexDocs.md

# in the next step we use pdflatex to convert .tex to .pdf
# of course we could use pandoc for this also, but pdflatex generates nicer PDFs
  - pdflatex -interaction=nonstopmode HowtoDeployLatexDocs.tex

# now we convert the pdf file to a much smaller djvu document  
  - pdf2djvu -d 1200 -o HowtoDeployLatexDocs.djvu HowtoDeployLatexDocs.pdf
after_success:
# upload the previously generated pdf file to simonwaldherr.de (use $UPLOADSECRET as password)
#  - wget --post-file=HowtoDeployLatexDocs.pdf -q -O- "https://simonwaldherr.de/pdf/?repo=HowtoDeployLatexDocs&mode=save&key=$UPLOADSECRET&pull=$TRAVIS_PULL_REQUEST"
deploy:
  provider: releases
  api_key:
    secure: P1+zIRw4qIH6BDHqRhZO80fAiXfpHaH58IW61fEZdR1i2cyHdq8tik5X2TZtfSNL0rDtNl8n2fV0Smy+MpJ4xYrWVi0ZYJ0xVLYJXz2HJPic6OBl7hgzm4fj28FqJ6j+glGwDIAH1LS1Rs2Jh3Kp18cuF8nklvoMbTfz1OY7JM1QkarwnnR0q2xw38ha1JRAAHB0OGzsqkIoNPdklhXd7toEuoap+9lhh8t2vFtr7rI/Eyup6NyWvHA4zEQShnrdQ92gobnqEDFEYx2hQTmLNlg1IMYioJqHysG8DfC4MT6AHMZwlB6o2eDRxA7BWd/WV7p8al7FaTbbnlo6PBRc/R8N8pKOVDlN9Jh+Rdrx0dLDpwjYKM6+Uu0wA0/to4A0qyp8gvekZYg63xbDUIcQ6sRc4SnA/8+/QogUu7HaEc9bdAEDYUNzxz6RGC05dkG9eXvlFznfLsOdCk3Zsde8OKGivY/oQEKcvdV5vNo5OPTrqHhORKzP2xGTptkiPbIzOHzCx0/V85mydBFVXBvARKJkGFDrVgX+BoFJGfXeGGrSlxgYhs5CQD852mJAdJR/2b8xpgSXCQj9HwnUjhjsQxvIMlU7pKM/iL1+UdQpEV2uLI0z9zeQUrOfQlmnHZzBn14TOjRv0uIJLpudGWzpJXje0l1zwP5OwwLckSXxpRU=
  file: 
    - "HowtoDeployLatexDocs.pdf"
    - "HowtoDeployLatexDocs.djvu"
  skip_cleanup: true
  on:
    repo: SimonWaldherr/testrepo
\end{verbatim}

\subsection{Rechtliches}\label{rechtliches}

Alle Angaben ohne Gewähr. Keine Garantie für Vollständigkeit oder
Aktualität. Jeglicher Inhalt dieses Repos ist lizenziert unter einer
\href{http://creativecommons.org/licenses/by-sa/4.0/}{Creative Commons
Namensnennung - Weitergabe unter gleichen Bedingungen 4.0 International
Lizenz}.

\end{document}
