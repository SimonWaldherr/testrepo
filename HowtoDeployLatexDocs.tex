\documentclass[]{article}
\usepackage{lmodern}
\usepackage{amssymb,amsmath}
\usepackage{ifxetex,ifluatex}
\usepackage{fixltx2e} % provides \textsubscript
\ifnum 0\ifxetex 1\fi\ifluatex 1\fi=0 % if pdftex
  \usepackage[T1]{fontenc}
  \usepackage[utf8]{inputenc}
\else % if luatex or xelatex
  \ifxetex
    \usepackage{mathspec}
    \usepackage{xltxtra,xunicode}
  \else
    \usepackage{fontspec}
  \fi
  \defaultfontfeatures{Mapping=tex-text,Scale=MatchLowercase}
  \newcommand{\euro}{€}
\fi
% use upquote if available, for straight quotes in verbatim environments
\IfFileExists{upquote.sty}{\usepackage{upquote}}{}
% use microtype if available
\IfFileExists{microtype.sty}{%
\usepackage{microtype}
\UseMicrotypeSet[protrusion]{basicmath} % disable protrusion for tt fonts
}{}
\ifxetex
  \usepackage[setpagesize=false, % page size defined by xetex
              unicode=false, % unicode breaks when used with xetex
              xetex]{hyperref}
\else
  \usepackage[unicode=true]{hyperref}
\fi
\usepackage[usenames,dvipsnames]{color}
\hypersetup{breaklinks=true,
            bookmarks=true,
            pdfauthor={},
            pdftitle={},
            colorlinks=true,
            citecolor=blue,
            urlcolor=blue,
            linkcolor=magenta,
            pdfborder={0 0 0}}
\urlstyle{same}  % don't use monospace font for urls
\setlength{\parindent}{0pt}
\setlength{\parskip}{6pt plus 2pt minus 1pt}
\setlength{\emergencystretch}{3em}  % prevent overfull lines
\providecommand{\tightlist}{%
  \setlength{\itemsep}{0pt}\setlength{\parskip}{0pt}}
\setcounter{secnumdepth}{0}

\date{}

% Redefines (sub)paragraphs to behave more like sections
\ifx\paragraph\undefined\else
\let\oldparagraph\paragraph
\renewcommand{\paragraph}[1]{\oldparagraph{#1}\mbox{}}
\fi
\ifx\subparagraph\undefined\else
\let\oldsubparagraph\subparagraph
\renewcommand{\subparagraph}[1]{\oldsubparagraph{#1}\mbox{}}
\fi

\begin{document}

\section{Veröffentlichen von LaTeX Dokumenten als PDF mittels GitHub und
TravisCI}\label{veruxf6ffentlichen-von-latex-dokumenten-als-pdf-mittels-github-und-travisci}

Dieses Dokument soll erklären, wie man mit den Diensten
\href{https://github.com/}{GitHub} und
\href{https://travis-ci.org/}{TravisCI} LaTeX Dokumente automatisch in
PDFs umzuwandeln und veröffentlichen kann. Der Vorteil dieses Verfahrens
gegenüber der lokalen Umwandlung (bspw. via git hooks) und Inkludierung
in das Repository ist u.a.:

\begin{itemize}
\tightlist
\item
  keine Binärdaten im Repo
\item
  Prüfung des Codes bei jedem Commit
\item
  Prüfung von Pull Requests
\item
  Unabhängigkeit vom lokalem System
\end{itemize}

Natürlich lässt sich dies auch um die Komponente \textbf{pandoc}
erweitern um somit auch die Umwandlung von Markdown abzudecken.

In der Anleitung gehe ich davon aus, dass es sich um ein öffentliches
Dokument handelt. Somit können Gratis Accounts bei
\href{https://github.com/}{GitHub} und
\href{https://travis-ci.org/}{TravisCI} verwendet werden. Es entstehen
keine Kosten durch die Nutzung.

\subsection{Schritt für Schritt}\label{schritt-fuxfcr-schritt}

\begin{enumerate}
\def\labelenumi{\arabic{enumi}.}
\tightlist
\item
  Erstellung eines git-Repositories

  \begin{itemize}
  \tightlist
  \item
    \href{https://github.com/SimonWaldherr/testrepo\#fork-destination-box}{dieses
    Repo forken} \emph{oder}
  \item
    \href{https://github.com/new}{neues Repo auf GitHub erstellen}
  \end{itemize}
\item
  \texttt{git\ clone} das Repo (\emph{ja, man könnte auch online
  editieren, aber das ignorieren wir mal})
\item
  zur Konfiguration benutzen wir die .travis.yml-YAML-Datei
\end{enumerate}

\subsection{Travis Konfig (YAML)}\label{travis-konfig-yaml}

\begin{verbatim}
foobar
lorem ipsum
\end{verbatim}

\subsection{Rechtliches}\label{rechtliches}

Alle Angaben ohne Gewähr. Keine Garantie für Vollständigkeit oder
Aktualität. Jeglicher Inhalt dieses Repos ist lizenziert unter einer
\href{http://creativecommons.org/licenses/by-sa/4.0/}{Creative Commons
Namensnennung - Weitergabe unter gleichen Bedingungen 4.0 International
Lizenz}.

\end{document}
